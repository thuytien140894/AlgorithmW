Type systems are fundamental to the study and development of programming languages. 
With the presence of explicit type annotations in a given program, a type system can deduce 
the final type or detect errors using a series of type rules applicable to each construct 
of a language. On the other hand, types can be inferred, without the need of any annotations, through constraints 
on the type domain of an expression based on its primitive value or its semantics. For instance, 
an integer is automatically assigned to type \texttt{Int}, whereas an argument to a function of type 
$\texttt{Int} \rightarrow \texttt{Bool}$ must have type $\texttt{Int}$.

Hindley-Milner type inference supports parametric polymorphism in the $\lambda$-calculus. There are no type 
annotations, and an operation can accept all types of values. Besides type constants, the Hindley-Milner 
system uses type variables to represent all possible type instantiations for an untyped operation. 
In this paper, we present an implementation of the Hindley-Milner type inference system using the 
linear-time algorithm W. The organization of the paper is as follows. First, we review the language syntax. 
We then outline some background literature on Hindley-Milner type inference. Finally, 
we describe the implementation details of type inference in our language. The last section 
concludes our work.