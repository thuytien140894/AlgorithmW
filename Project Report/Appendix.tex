\section{Lexer}
The library \texttt{Parsec} is used to implement the lexer and parser. 
We first create a language definition specifying how individual characters are tokenized, as follows:
\begin{lstlisting}
    langDef :: Tok.LanguageDef ()
    langDef = Tok.LanguageDef
        { Tok.commentStart    = ""  
        , Tok.commentEnd      = ""
        , Tok.commentLine     = "//"
        , Tok.nestedComments  = False
        , Tok.identStart      = letter 
        , Tok.identLetter     = alphaNum 
        , Tok.opStart         = oneOf ":!#$%&*+./<=>?@\\^|-~"
        , Tok.opLetter        = oneOf ":!#$%&*+./<=>?@\\^|-~"
        , Tok.reservedNames   = [ "true"
                                , "false"
                                , "if"
                                , "then"
                                , "else"
                                , "succ"
                                , "pred"
                                , "iszero"
                                , "zero"
                                ]
        , Tok.reservedOpNames = [ "succ"
                                  , "pred"
                                  , "iszero" 
                                  ]
        , Tok.caseSensitive   = True
        }
\end{lstlisting}
and then initialize a lexer using this language definition.
\begin{lstlisting}
    lexer :: Tok.TokenParser ()
    lexer = Tok.makeTokenParser langDef
\end{lstlisting}

\section{Parser}
The parser uses the tokens provided by the lexer to construct a valid abstract 
syntax tree for the input program. We first define separate parsers for each expression type. Parsing constants 
and variables is straightforward. 
\begin{lstlisting}
    -- | Parse a variable.
    var :: Parser Expr
    var = Var <$> identifier 

    -- | Parse constants.
    true, false, zero :: Parser Term
    true  = reserved "true" >> return Tru
    false = reserved "false" >> return Fls
    zero  = reserved "0" >> return Zero
\end{lstlisting}
We parse arithmetic and boolean operators as unary prefix operators.
\begin{lstlisting}
    prefixTable :: Ex.OperatorTable String () Identity Term
    prefixTable = 
        [ [ Ex.Prefix $ reserved "succ"   >> return Succ
          , Ex.Prefix $ reserved "pred"   >> return Pred
          , Ex.Prefix $ reserved "iszero" >> return IsZero
          ]
        ]
\end{lstlisting}
The following are the parsers for conditionals, $\lambda$-abstractions, and \texttt{let}-expressions.
\begin{lstlisting}
    -- | Parse an if statement.
    conditional :: Parser Expr
    conditional = do
        reserved "if"
        cond <- expr
        reserved "then"
        tr <- expr
        reserved "else"
        fl <- expr
        return $ If cond tr fl

    -- | Parse an abstraction.
    lambda :: Parser Expr
    lambda = do
        reservedOp "\\" 
        x <- identifier 
        dot 
        body <- expr
        return $ Lambda x body

    -- | Parse a let expression.
    letExpr :: Parser Expr 
    letExpr = do 
        reserved "let"
        x <- identifier
        reservedOp "=" 
        val <- expr
        reserved "in"
        body <- expr
        return $ Let x val body
\end{lstlisting}

At the top level, the parser parses an expression as an application by first parsing one or 
more simple expressions (non-applications) separated by a space and then applying them from left to right.
\begin{lstlisting}
    app :: Parser Term
    app = do
        terms <- sepBy1 expr' whiteSpace 
        return $ applyFromLeft terms
\end{lstlisting}

\section{Pretty Printer}
The library \texttt{PrettyPrint} is used to implement a printer, which converts expressions from their AST representations to 
user-readable formats. We create separate printers for expressions, types, and errors using  
the typeclass \lstinline{Pretty}. 
\begin{lstlisting}
    class Pretty a where 
        output :: a -> Doc
        
        printMsg :: a -> IO ()
        printMsg = PP.putDoc . output
\end{lstlisting}
The \lstinline{output} function converts a type $a$ to a 
\texttt{Doc}, which is a set of layouts. This function is left as abstract because each \texttt{Pretty} instance  
requires different formatting styles. \texttt{printMsg} has a default implementation of 
piping the result of \texttt{output} to \texttt{PP.putDoc} in order to print type $a$ to the standard output. 
This function is shared by all the \texttt{Pretty} instances. 