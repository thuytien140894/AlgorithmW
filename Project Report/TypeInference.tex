Figure 1 presents the inference rules based on the type system of our language 
using the judgment
\begin{align*}
    \Gamma \vdash e : \sigma
\end{align*}
where $\Gamma$ is the type environment mapping each bound variable to its type scheme. 
In the context of type inference, $\Gamma$ serves more as a set of type assumptions. 
These rules can be found in Damas and Milner \cite{damas} with some additional rules 
for arithmetic, boolean, and conditional expressions. Rules {\scriptsize{[TAUT]}}  
and {\scriptsize{[CONST]}} used to infer the type of a variable and a constant   
are straightforward. Rules {\scriptsize{[SUCC]}}, 
{\scriptsize{[PRED]}}, and {\scriptsize{[ISZERO]}} enforce a \texttt{Nat} argument 
for arithmetic and boolean operators. Provided a boolean condition and equivalent branch types, 
the type of a conditional expression resolves to its branch type, as 
stated in the {\scriptsize{[IF]}} rule. Rule {\scriptsize{[COMB]}} infers the type 
for an application whenever the argument type matches the parameter type of the operator. 
The type of a $\lambda$-abstraction is simply a mapping between the argument type and 
the type of the body expression. We infer the type of a \texttt{let}-expression under the 
assumption that all the $x$ occurrences in the binding context have the type of 
the expression to which $x$ is assigned initially. The symbol $\Gamma _x$ used in 
rules {\scriptsize{[ABS]}} and {\scriptsize{[LET]}} denotes the removal of 
any type assumption for variable $x$ from the type environment $\Gamma$.
\begin{figure}[ht]
    \caption{$\lambda ^? _{\rightarrow}$ Type Inference Rules}
    \hrule
    \[
        \begin{bprooftree}
            \AxiomC{$(x\!:\!\sigma) \in \Gamma$}
            \RightLabel{\scriptsize{\text{[TAUT]}}}
            \UnaryInfC{$\Gamma \vdash x : \sigma$}
        \end{bprooftree}
        \begin{bprooftree}
            \AxiomC{$\Delta k = \tau$}
            \RightLabel{\scriptsize{\text{[CONST]}}}
            \UnaryInfC{$\Gamma \vdash k : \tau$}
        \end{bprooftree}
        \begin{bprooftree}
            \AxiomC{$\Gamma \vdash e : \sigma$}
            \AxiomC{$\sigma > \sigma'$}
            \RightLabel{\scriptsize{\text{[INST]}}}
            \BinaryInfC{$\Gamma \vdash e : \sigma'$}
    \end{bprooftree}
    \]
    \vspace{1mm}
    \[
        \begin{bprooftree}
                \AxiomC{$\Gamma \vdash e : \sigma$}
                \AxiomC{$\overrightarrow{\alpha} \not \in dom(\Gamma)$}
                \RightLabel{\scriptsize{\text{[GEN]}}}
                \BinaryInfC{$\Gamma \vdash e : \forall \overrightarrow{\alpha}. \: \sigma$}
        \end{bprooftree}
        \begin{bprooftree}
            \AxiomC{$\Gamma \vdash e : \texttt{Nat}$}
            \RightLabel{\scriptsize{\text{[SUCC]}}}
            \UnaryInfC{$\Gamma \vdash \texttt{succ} \: e : \texttt{Nat}$}
        \end{bprooftree}
    \]
    \vspace{1mm}
    \[
        \begin{bprooftree}
            \AxiomC{$\Gamma \vdash e : \texttt{Nat}$}
            \RightLabel{\scriptsize{\text{[PRED]}}}
            \UnaryInfC{$\Gamma \vdash \texttt{pred} \: e : \texttt{Nat}$}
        \end{bprooftree}
        \begin{bprooftree}
            \AxiomC{$\Gamma \vdash e : \texttt{Nat}$}
            \RightLabel{\scriptsize{\text{[ISZERO]}}}
            \UnaryInfC{$\Gamma \vdash \texttt{iszero} \: e : \texttt{Bool}$}
        \end{bprooftree}
    \]
    \vspace{1mm}
    \[
        \begin{bprooftree}
                \AxiomC{$\Gamma \vdash e_1 : \texttt{Bool}$}
                \AxiomC{$\Gamma \vdash e_2 : \tau$}
                \AxiomC{$\Gamma \vdash e_3 : \tau$}
                \RightLabel{\scriptsize{\text{[IF]}}}
                \TrinaryInfC{$\Gamma \vdash \texttt{if} \: e_1 \: \texttt{then} \: e_2 \: 
                \texttt{else} \: e_3 : \: \tau$}
        \end{bprooftree}
    \]
    \vspace{1mm}
    \[
        \begin{bprooftree}
            \AxiomC{$\Gamma \vdash e_1 : \tau' \rightarrow \tau$}
            \AxiomC{$\Gamma \vdash e_2 : \tau'$}
            \RightLabel{\scriptsize{\text{[COMB]}}}
            \BinaryInfC{$\Gamma \vdash e_1 \: e_2 : \tau$}
        \end{bprooftree}
        \begin{bprooftree}
            \AxiomC{$\Gamma _x \cup \{x\!:\!\tau'\} \vdash e : \tau$}
            \RightLabel{\scriptsize{\text{[ABS]}}}
            \UnaryInfC{$\Gamma \vdash \lambda x. \: e : \tau' \rightarrow \tau$}
        \end{bprooftree}
    \]
    \vspace{1mm}
    \[
        \begin{bprooftree}
            \AxiomC{$\Gamma \vdash e : \sigma$}
            \AxiomC{$\Gamma _x \cup \{x\!:\!\sigma\} \vdash e' : \tau$}
            \RightLabel{\scriptsize{\text{[LET]}}}
            \BinaryInfC{$\Gamma \vdash (\texttt{let} \: x = e \: \texttt{in} \: e') : \tau$}
        \end{bprooftree}
    \]
    \hrule
    \end{figure}  

The remaining two rules {\scriptsize{[INST]}} and {\scriptsize{[GEN]}} capture 
the notion of type instantiation and type closure, respectively. An instantiation 
of a type scheme $\sigma$ replaces all the quantified type variables $\alpha _i$ in $\sigma$ 
with fresh type variables $\beta _i$ \cite{heeran}
\begin{align*}  
    instantiate(\forall \alpha _1,...,\alpha _n. \: \sigma) = \sigma[\alpha _1 \mapsto \beta _1,...,
    \alpha _n \mapsto \beta _n]
\end{align*}
The resulting type scheme $\sigma'$ is less general than $\sigma$, written as $\sigma > \sigma'$. 
On the other hand, the closure of type $\tau$ with respect to a type environment 
$\Gamma$ is denoted as 
\begin{align*}
    \overline{\Gamma}(\tau) = \forall \alpha _1,...,\alpha _n. \: \tau
\end{align*}
where $\alpha _1,...,\alpha _n$ are free in $\tau$ but not in $\Gamma$ \cite{damas}. Observe that 
while type instantiation removes universal quantifiers for type variables, type closure 
adds them to a type scheme.